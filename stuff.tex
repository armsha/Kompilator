\section{Användarhandledning}
	\label{Sec:anv}

	%Plats
	Uppgiftens lösning kan återfinnas på följande plats under cs servrar:
\begin{verbatim}
/home/c12/c12oor/edu/dv3/lab/
\end{verbatim}
	De intressanta filerna är de följande;
\begin{verbatim}
lab2*         // Den körbara kompilatorn.
lab.y         // Huvuddelen av programmet.
lab.h         // Konstanter och typer använda.
lab.l         // Lexfil använd.
Makefile      // Kompilerar och bygger allt i ordning.
stack.c (.h)  // Datatyp använd
table.c       // Hantering av symboler och kommandon med mera.
\end{verbatim}

	%Make
	För att bygga den körbara kompilatorn kan \textit{make} användas i en mapp med alla nödvändiga filer. Denna kommer att utföra alla mellansteg i ordning.
	
	%körning
	Då filen har byggds korrekt kan man sedan köra den. Den tar ingen indata vid uppstart utan väntar på att text matas på standard-input. Den läser sedan tills texten tar slut och matar sedan ut tolkningen av symboler och den genererade mellankoden.	
	
	%Fel?
	Om någonting inte stämmer med den kod man matat in kommer fel rapporteras, och programmet antingen avslutas direkt, eller fortsätta om endast mindre fel ( som en odeklarerad variabel ) upptäckts.

\pagebreak

\section{ Ändringar och Justeringar }
	För att få mellankodsgenereraren att fungera som önskat har ett antal 
	element ur de olika tillhandahållna resurserna ändrats. Både i grammatiken
	och den grundläggande koden, men utöver detta har ett antal förändringar av hur
	resultatet ser ut också gjorts.

	\subsection{Grammatik}
		Dels har ett par regler lagts till i grammatiken motsvarande utökningar
		på den grundläggande funktionaliteten. Utöver detta har även ett antal regler 
		modifierats för att undvika konflikter.
		
		% Ntok, Mark
		\subsubsection{Markörer}
			För att kunna hålla reda på de kodpositioner till vilka man behöver hoppa eller
			förflytta sig användes ett par markörer.

			\begin{description}
				\item[Ntok]	 Markerar ett goto-kommando och ger en lista vilken kan användas för att peka en serie andra delar till kommandot.
				\item[Mark1] Håller koll på platsen i koden som efterföljer den nuvarande.
			\end{description}

			Dessa markörer läggs sedan till i kontrollstrukturer och loopar för att generera korrekt mellankod, med hopp till rätt platser. Till exempel ser en if-else-sats ut som följer med markörer: 
			
\begin{verbatim}
Stat :IFSY expr THENSY Mark1 stat Ntok ELSESY Mark1 stat
\end{verbatim}

			Med backpatching utfört genererar detta mellankod som ser ut som i tabell \ref{Tab:if}.
			
\begin{table}[!htbp]
	\centering
	\begin{tabular}{ l | c | p{.17\textwidth} | p{.35\textwidth} }
\textbf{Position} & \textbf{Kommando} & \textbf{Efterföljande Position} & \textbf{Kommentar} \\ \hline
i:  &     <expr>   &  i+2 & Jämförelsen görs på rad i. (Gå till nästa rad om falsk, annars följs addressen. ) \\
i+1:   &  GOTO    &   i+j+3 &      Hoppa över första och gör andra. \\
i+2:   &  <stat1>  &  ? 	&     Sant utför första stat (längd j). \\
... &&&\\
i+j+2:  & GOTO    &   i+j+k+3  & Hoppa till fortsättning då stat är klar. \\
i+j+3:  & <stat2>  &  ?      &   Utför andra stat (längd k). \\
i+j+k+3:& <forts>  &  ?      &   Fortsätt program efter. \\
	\end{tabular}
	\caption{Mellankod för if-sats efter backpatching av addresser till gotosatser och så vidare.}
	\label{Tab:if}
\end{table}
		
			Här har markeringar och efterföljarlistor använts för att ge goto-kommandona sina adresser, då
			dessa inte var kända medans de lästes, utan endast kunnat utrönas då kommandona i de båda stat
			laggts till.


		% Repeatsy
		\subsubsection{Repeat-Until}
			Som en del av uppgiften skulle en ny regel för ''upprepa tills uttrycket uppfylls'' läggas till.
			Denna regel ser ut som följer:
\begin{verbatim}
stat  ->  repeat stat until expr ; 	
\end{verbatim}
			Med markörer blir detta:
\begin{verbatim}
stat  : REPEATSY Mark1 stat UNTILSY Mark1 expr SEMI  ; 	
\end{verbatim}

			Då backpatching och så vidare hade tagits hand om för resten av koden krävdes sedan endast tilläggning av följande rader för att ta hand om denna regel.

\begin{verbatim}
backpatch( $6.falselist, $2 ); 	// Backpatch expr false to goto the first mark
backpatch( $3, $5 );            // Backpatch stat nextlist to mark before expr

$$ = $6.truelist;               // Return nextlist that is the truelist of expr
\end{verbatim}

			Detta utför korrekt funktionaliteten som upprepar uttrycket \textit{stat} genom att hoppa till den första markeringen så länge som \textit{expr} är falskt, samt fortsätta jämföringen och loopningen ända tills den blir sann.

		% Ident/fname kollision
		\subsubsection{Kollisioner}
			Då grammatiken testades med de symboler och regler som givits upptäcktes kollisioner. Programmet kunde inte se skillnad mellan två olika regler utefter vad som den läst. Dessa regler var de följande:
\begin{verbatim}
stat    -> ...
        | id assop expr ;
        | fname assop expr ;   // Removed bc collision, handled as id special case
\end{verbatim}

			Genom att ta bort en av dem, och istället utföra bestämmelsen av hurvida det som tilldelades var en variabel eller ett funktionsnamn med hjälp av en if-sats på identifierarnamnets typ kunde denna kollision undvikas.

	% Flyttal
	\subsection{Flyttal}
			För att undersöka funktionaliteten hos typhantering i grammatiken lades ytterligare en typ till. Det existerar alltså numera utöver \textit{integer} även en typ \textit{float}, vilken kan identifieras med nyckelordet ''float'' i deklarationslistor, eller som <siffror.siffror> vid definition av konstanter.

			Då ingen regel, symbol, tolkning, eller hantering av denna typ fanns har ändringar alltså gjorts i såväl token-, och regel-definitionen i y-filen, samt tolkning av nyckelord och konstant-uttryck i lex-filen. Även table.c och lab.h har ändrats för att kunna identifiera och skriva ut denna nya typ.

			Genom att lägga till den nya typen visas styrkan i kompilatorns typhantering. Det blir enkelt att utöka med fler grundläggande typer, utan att nya regler och hanteringar behöver läggas till, förutom de som tar hand om identifiering.

	\subsection{Symboltabell och Åtkomst}
	
		En del ändringar har gjorts angående hur de returnerade symbol och kodtabellerna ser ut, samt även direkta förändingar på hur språket fungerar. Dessa kan i vissa fall få tidigare korrekta program att inte längre fungera som de ska.
	
		% Globala variablerr
		\subsubsection{Globala Variabler}
		
			En ändring som utförts var på hurvida globala variabler, eller variabler deklarerade utanför det nuvarande namnrummet, skulle få användas. Då detta leder till läsbarhetsproblem utan att ge mer användbarhet än några möjligheter till enklare sidoeffekter i funktioner, beslutades det att detta fall skulle tas bort.
			
			Det är alltså inte längre lagligt att använda variabler deklarerade i yttre funktioner, utan allt som används måste nu mera antingen passeras som argument till den anropade funktionen, eller deklareras på nytt.
			
			Ett vanligt problem att kolla för problem mellan de olika versionerna blir alltså \textit{identifier not declared}, vilket dyker upp då man inte deklarerat en variabel korrekt.
		
		% Rekursion
		\subsubsection{Rekursion}
		
			För att öka styrkan hos programspråket, och göra det enkelt och elegant att utföra ett flertal uppgifter har rekursion tillåtits. Detta innebär att funktioner kan kallas även från funktionen själv.
			
			Denna typ av upprepning är ofta använd och ganska enkel att överskoda. Förutom denna fördel är det även enkelt att lägga till och ta bort, då allt som krävs är att symboltabellen för funktionen får en post till med funktionen själv tillagd vid deklarationen.
			
		
		% Main Scope
		\subsubsection{Main Scope}
		
			Vid utskrift av symboltabell har vi valt att även mata ut det sista namnrummets symboler under ''Main Scope:''. Denna innehåller i de tillhandahållna programmen endast programmet på nivå (\textit{level}) 0.
			
			Detta görs då det annars är oklart vad nivåer representerar, samt även för att tillåta ett flertal program och så vidare i samma textfil.

		\subsubsection{Konsekvenser}
			\label{Sec:cha}
			%ok4.prg
			
			De flesta program, även korrekta sådana, genererar efter dessa ändringar annan mellankod. Främst ser symboltabeller annorlunda ut från tidigare, då funktionsnamn följer med i funktionen också, samt utskrifter av ytterligare ett namnrum.
			
			Att notera är dock att vissa program som tidigare varit otillåtna nu är tillåtna, och vice versa. Rekursiva funktioner räknas till exempel numera som acceptabla, medans variabler använda från yttre namnrum ger felmeddelanden.
			
			Specifikt är testprogrammet \textbf{ok4.prg} inte längre korrekt då variabler \textit{f1v2}, \textit{v2}, och \textit{f2v2} på vissa ställen inte deklareras i de funktioner som de används.
			

	\subsection{Ändringar av Kod}
		% Nya typer, stack-idlist-scope
		% destroy och andra ändringar
		% Tillagt och ändrat för repeat och bra utskrifter

		Även delar av den tilhandahållna hjälpkoden har modifierats, tagits bort, eller utökats. Bland annat funktioner för att rensa nuvarande symboltabell laggts till, och den är även tillgänglig för andra filer så att den kan sparas undan för att plockas fram senare. Även nya egna datatyper har laggts till, till exempel listor för identifierare, och stackar för namnrum. Mer om dessa ändringar och utökningar under systembeskrivningen.

\pagebreak

\section{Systembeskrivning}

	% Fil-för fil?
	% Regel för regel?
	% Backpatching, Scopning, Typning, speciellt och generellt
	% Datatyper, och hanterning
	% Kommenterad kod.
	
	Systemet består av ett antal filer, listade under sektion \ref{Sec:anv}, Användarhandledning. Dessa krävs alla för att kunna bygga programmet, och ansvarar vardera för någon deluppgift under kompileringen. I denna sektion beskrivs systemets uppbyggnad samt hur de olika delarna tas omhand.
	
	Mer information om specifika delar av programmet kan återfinnas under de kommentarer som lämnats i koden.	
	
	\subsection{Lexning}
		För att kunna läsa en textfil och identifiera olika ord, specialtecken, och numeriska värden krävs vad som kallas en \textit{lexikalanalysator}. Denna tolkar i vårt fall text med hjälp utav reguljära uttryck, och skickar sedan tolkningen vidare i form av \textit{tokens}, samt värden tillhörande dessa. Den läser alltså tecken för tecken och skickar vidare de tokens den hittar till kompilatorn.
		
		Ett exempel är tolkningen av ett heltal, vilket återfinns med ett reguljärt uttryck som accepterar en eller flera siffror, och sedan skickar vidare tokenet ''INTCONST'', samt värdet i form av siffrorna i textformat.
		
		För att kunna utföra sitt arbete använder lexikalanalysatorn, återfunnen i \textit{lab.l}, även en tabell över nyckelord, söfunktioner, och lite felhantering.
	
	\subsection{Parsning}	
		Då token skickats korrekt ska de sedan tolkas enligt de grammatiska regler som språket följer. Detta görs utav filen \textit{lab.y}, i vilken grammatiken matats in, tillsammans med mellankodsgenerering.
		
		Tolkningen efter reglerna görs troligtvis via en algoritm lik någon LR-parsning \footnote{Se referenser till DV3 slides.}. För att detta ska fungera krävs att reglerna är tillräckligt olika, varför ett par regler har ändrats lite från grundgrammatiken. Även tokens vilka inte motsvarar någon text har lagts till som markörer för hjälp med mellankodsgenereringen.
		
		Korrekt mellankod genereras genom tilläggande av c-kod på strategiska punkter under reglernas, och denna körs på den  punkt den återfinns i regeln då den tolkas. Mer om mellankodsgenereringen kan återfinnas i följande sektion (\ref{Sec:mel}).
		
	\subsection{Mellankod}
		\label{Sec:mel}
		
		Mellankoden som genereras består dels utav tabeller med symboler, med all information dessa behöver. Samt mellankod med korrekta addresser, symboler och kommandon.
		
		Genereringen av denna sker i kod tillagd i regler och specialsektioner i \textit{lab.y}, men ett flertal andra filer för datastrukturer, utskrift, och så vidare, används också. Att notera är \textit{table.c}, samt \textit{stack.c}.
		
	\subsubsection{Argumentlistor}
		I till exempel anrop till ''read'' tas ett flertal argument in. För att kunna hantera alla dessa används listor som byggs upp argument för argument. Till detta ändamål har en simpel, enkel-länkad lista implementerats, kallad ''idlist''. Denna används då ett flertal identifierare behöver läsas in, som i read-funktione, eller i deklarationer.
		
		Även en implementation av datatypen stack har gjorts (även den enkel-länkad). Denna används bland annat vid uppsamlandet av ett flertal uttryck, som vid anrop till write.
		
		Det krävs insamling av värden i dessa fall, då reglerna för uttrycks- eller id-listor inte känner till mycket nog om vad värdena ska användas till, samt att de används till ett flertal olika saker.
	
	\subsubsection{Typhantering}
		Typhanteringen byggdes genom att alltid kolla på värdet som \textit{type}-tokenet tilldelats, även fast det i ursprungsversionen inte fanns fler typer än en. Det krävdes även lite smart strukturering av koden i reglerna så att typen hade lästs in innan den användes.
		
		För att kunna hantera de många olika reglerna krävdes det att ett antal olika typer ( listor, stackar, heltal, och strängar med mera ) kunde skickas som värden med tokens. Detta styrdes med .y-filens \%union.
	
	\subsubsection{Namnrumshantering}
		Hanteringen av symboltabeller gjordes via table.c's 	SYMBOL hanteringsfunktioner. Den innehåller funktioner för att kunna lägga till, kolla upp och skriva ut sådanna SYMBOLer. För att kunna använda dem enkelt utfördes dock ett par ändringer. Dels så att variablen symbtab kan nås även från parsaren, och därigenom användas för att spara ett namnrum för senare användning, och dels tilläggandet av en funktion för att ta bort en symboltabell.
		
		Även stackdatatypen används vid hanteringen av symboler. Då man stegar in i en ny funktion vill man ju ha kvar den yttre funktionens symboler så att de kan användas när den inre funktionen är klar. Detta plus det faktum att inget ur de yttre funktionerna får användas i de inre tillät oss att hålla de äldre namnrummen på stacken och sedan plocka av dem då de innre är klara. Med ett par extrafunktioner som kallas varje gång man går in och ut ur nya funktioner kan symboltabeller hållas i en stack för varje nivå.
	
	\subsubsection{Backpatching}
		Då koden parsas krävs det ibland att kommandon för hopp till andra rader i koden behöver läggas till. En sådan situation är till exempel i if-satser, där koden ska fortsätta från olika platser beroende på om ett uttryck är sant eller falskt. Detta utförs med ett GOTO kommando i mellankoden, men ett problem uppstår då dessa hopp ska göras frammåt. Addressen för den position till vilket hoppet ska göras har då inte lästs in ännu när GOTO-kommandot matas ut. För att fixa detta används så kallad backpatching\footnote{Se referens till Drakboken, \textit{Compilers: Principles, Techniques, \& Tools}.}.
		
		Markörer läggs till i de grammatiska reglerna på strategiska platser. Dels markörer som håller reda på vilken plats i mellankoden som återfinns på ett visst ställe i regeln, och dels markörer som håller reda på en viss GOTO-sats och kan användas för att i efterhand gå igenom dessa och ändra måladressen till de korrekta platserna. Typen som används för att kunna gå tillbaka och ändra måladresser för nästa kodrad att köra efter kommandon kallas nextlist, och de kan innehålla ett antal poster med kodrader, vilka alla kommer att pekas mot det nya målet vid backpatchning. 
		
		De flesta av reglerna som kallas stat(ments) har hand om sådanna nextlistor, och backpatchar, mergar och skapar nya sådanna då det behövs. Till exempel för att ta hand om alla de loopar och hopp som behöver göras i kontrollstrukturerna.
		
		För att ta hand om jämförelseuttryck används även typen expression, vilken har hand om två nextlistor (och en symbol). En av dessa nextlistor används för fallet då uttrycket var sant, och pekar till jämförelseradens måladdress. Det andra, för då uttrycket var falskt, pekar till en gotosats som hoppar över den kod som körs om det hade varit sant. 

\pagebreak