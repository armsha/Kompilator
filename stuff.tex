

\section{Ändringar och Justeringar}
	För att få mellankodsgenereraren att fungera som önskat har ett antal 
	element ur de olika tillhandahållna resurserna ändrats. Både i grammatiken
	och den grundläggande koden, men även ett antal förändringar av hur
	resultatet ser ut har gjorts.

	\subsection{Grammatik}
		Dels har ett par regler lagts till i grammatiken motsvarande utökningar
		på den grundläggande funktionaliteten. Utöver detta har även ett antal regler 
		modifierats för att undvika konflikter.

		% Ntok, Mark
		\subsubsection{Markörer}
			För att kunna hålla reda på de kodpositioner till vilka man behöver hoppa eller
			förflytta sig användes ett par markörer.\\

			\begin{description}
				\item[Ntok]	 Markerar ett goto-kommando och ger en lista vilken kan användas för att peka en serie andra delar till kommandot.
				\item[Mark1] Håller koll på platsen i koden som efterföljer den nuvarande.
			\end{description}

			Dessa markörer läggs sedan till i kontrollstrukturer och loopar för att generera korrekt mellankod, med hopp till rätt platser. Till exempel ser en if-else-sats ut som följer med markörer: \\
\verbatim{stat :IFSY expr THENSY Mark1 stat Ntok ELSESY Mark1 stat}\\
			Med backpatching utfört genererar detta mellankod som ser ut som följer:\\
\begin{table}[h!]
	\begin{tabular}{ l | c | c | l }
\textbf{Position} & \textbf{Kommando} & \textbf{Efterföljande Position} & \textbf{Kommentar} \\ \hline
i:  &     <expr>   &  i+2 & Jämförelsen görs på rad i \\
i+1:   &  GOTO    &   i+j &      Hoppa över första och gör andra \\
i+2:   &  <stat1>  &  ? 	&     Sant utför första stat (längd j) \\
... &&&\\
i+j+2:  & GOTO    &   i+j+k+3  & Hoppa till fortsättning då stat är klar \\
i+j+3:  & <stat2>  &  ?      &   Utför andra stat (längd k) \\
i+j+k+3:& <forts>  &  ?      &   Fortsätt programm efter \\
	\end{tabular}
\end{table}
		
			Här har markeringar och efterföljarlistor använts för att ge goto-kommandona sina adresser, då
			dessa inte var kända medans de lästes, utan endast kunnat utrönas då kommandona i de båda stat
			laggts till.

		% Repeatsy
		\subsubsection{Repeat-Until}
			Som en del av uppgiften skulle en ny regel för "upprepa tills uttrycket uppfylls" läggas till.
			Denna regel ser ut som följer:\\
\verbatim{   stat  ->  repeat stat until expr ; 	}\\
			Med markörer 
		% Ident/fname kollision

		% Flyttal

	\subsection{Flyttal}

	\subsection{Symboltabell och Åtkomst}
		% Globala variablerr
		% Rekursion
		% Main Scope

		\subsubsection{Konsekvenser}

	\subsection{Ändringar av Kod}
		% Nya typer, stack-idlist-scope
		% destroy och andra ändringar
		% Tillagt och ändrat för repeat och bra utskrifter



\section{Systembeskrivning}
	\subsection{Backpatching}
